\usepackage{graphicx, bm, color, hyperref, txfonts, caption, physics, xltxtra, amsmath, amssymb, mathrsfs, comment, tikz}
% \usepackage[colorgrid,gridunit=pt,texcoord]{eso-pic}
% \mathtoolsset{showonlyrefs=true} % 参照したラベルのみ番号を付ける 07/07/18追加
\usepackage[version=4]{mhchem} %化学式
\usepackage[absolute,overlay]{textpos}

\usepackage{tcolorbox} % 箱 2020/08/28追加
\tcbuselibrary{breakable}
\newtcolorbox{mybox}[2][]{colbacktitle=orange,
  colback=orange!30!white,colframe=orange,
  title={#2},fonttitle=\bfseries\gtfamily,#1}

\def\seireki{ % 日付を西暦表示 (20xx/yy/zz) に。[11] を元に作成
  \the\year/
  \ifnum\month<10 0\fi\the\month/
  \ifnum\day<10 0\fi\the\day}

% オリジナルマクロ
\newcommand{\setz}{\mathbb{Z}}
\newcommand{\setn}{\mathbb{N}}
\newcommand{\setc}{\mathbb{C}}
\newcommand{\setq}{\mathbb{Q}}
\newcommand{\setr}{\mathbb{R}}
\newcommand{\pr}{^{\prime}}

\captionsetup{font=scriptsize}
\renewcommand{\figurename}{Fig}
\setmainfont{ヒラギノ角ゴシック W3}
\setsansfont{ヒラギノ角ゴシック W3}
\setmonofont{ヒラギノ角ゴシック W3}
\setromanfont{Times New Roman}
\XeTeXlinebreaklocale ``ja''
\usetheme[progressbar=frametitle]{metropolis}
\usefonttheme{professionalfonts}
\useinnertheme{rectangles}
\useoutertheme{default}
\usecolortheme{orchid}